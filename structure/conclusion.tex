% !TeX root = ../main.tex
结论与展望:结论包括对整个研究工作进行归纳和综合而得出的总结;所得结果与已有结果的
比较;联系实际结果,指出它的学术意义或应用价值和在实际中推广应用的可能性;在本课题研究
中尚存在的问题,对进一步开展研究的见解与建议。结论集中反映作者的研究成果,表达作者对所
研究课题的见解和主张,是全文的思想精髓,是全文的思想体现,一般应写得概括、篇幅较短。
一般致谢的内容有:

毕业论文的结论作为单独一章排列,但标题前不加“第 XXX 章”字样。

撰写时应注意下列事项:

\begin{enumerate}[label=(\chinese*),itemindent=2em]

    \item 结论要简单、明确。在措辞上应严密,但又容易被人领会。
    \item 结论应反映个人的研究工作,属于前人和他人已有过的结论可少提。
    \item 要实事求是地介绍自己研究的结果,切忌言过其实,在无充分把握时,应留有余地。

\end{enumerate}

本章节使用section等环境时,应带上*,不出现在目录中。即使用\clist{\section*{}}
\section*{总结}
\section*{展望}

% 使用以下命令手动控制空白页
\newpage
\indent
\newpage
